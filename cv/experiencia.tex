%-------------------------------------------------------------------------------
%	SECTION TITLE
%-------------------------------------------------------------------------------
\cvsection{Experiência}
%-------------------------------------------------------------------------------
%	CONTENT
%-------------------------------------------------------------------------------
\begin{cventries}

%---------------------------------------------------------
\cventry
	{Pesquisador Industrial II-01}
	{SENAI-PE - Instituto SENAI de Inovação em TICs}
	{Recife, Brasil}
	{Maio 2018 - Atual}
	{\begin{cvitems}
		\item{Pesquisa, Desenvolvimento e Inovação em Tecnologias da Informação e Comunicação para a Indústria}
	\end{cvitems}}

%---------------------------------------------------------
%---------------------------------------------------------
\cventry
	{Co-fundador, Designer de Interação e Desenvolvedor}
	{Batebit Artesania Digital}
	{Recife, Brasil}
	{Fev. 2014 - Atual}
	{\begin{cvitems}
		\item{Empresa de Concepção e Desenvolvimento de Novas Interfaces para Expressão Artística}
		\item{Link: \href{http://batebit.cc}{http://batebit.cc}}
	\end{cvitems}}

%---------------------------------------------------------
%---------------------------------------------------------
\cventry
	{Pesquisador}
	{MusTIC}
	{Recife, Brazil}
	{Maio 2011 - Atual}
	{\begin{cvitems}
		\item{Grupo de Pesquisa do Centro de Informática}
		\item{UFPE}
		\item{Foco em Música, Tecnologia, Interatividade e Criatividade}
		\item{Professores coordenadores: Geber Ramalho e Giordano Cabral}
	\end{cvitems}}

%---------------------------------------------------------
%---------------------------------------------------------
\cventry
	{Professor}
	{CESAR.School Manaus}
	{Manaus, Brasil}
	{}
	{
	\begin{cvitems}
	    \item{Mestrado Profissional em Design}
	\end{cvitems}
	\begin{cvsubentries}
        \cvsubentry{}{}{}{}
        \cvsubentry{}{Turma MPD20182 - Módulo Cidades Inteligentes e Cidades Lúdicas}{Ago. 2018}{}
        \cvsubentry{}{Turma MPD20181 - Módulo Cidades Inteligentes e Cidades Lúdicas}{Mar. 2018}{}
        \cvsubentry{}{Turma MPD20171 - Módulo Cidades Inteligentes e Cidades Lúdicas}{Maio. 2017}{}
    \end{cvsubentries}
    }

%---------------------------------------------------------
%---------------------------------------------------------
\cventry
	{Professor}
	{CESAR.School Recife}
	{Recife, Brasil}
	{}
	{
	\begin{cvitems}
		\item{Mestrado Profissional em Design}
	\end{cvitems}
	\begin{cvsubentries}
        \cvsubentry{}{}{}{}
        \cvsubentry{}{Módulo Cidades Inteligentes e Cidades Lúdicas}{Out. 2017}{}
        \cvsubentry{}{Módulo Cidades Inteligentes e Cidades Lúdicas}{Set. 2016}{}
        \cvsubentry{}{Módulo Prototipação Eletrônica}{Nov. 2014}{}
    \end{cvsubentries}
    }

%---------------------------------------------------------
%---------------------------------------------------------
\cventry
	{Pesquisador}
	{Daccord Music Software}
	{Recife, Brazil}
	{Abr. 2017 - Mar. 2018}
	{\begin{cvitems}
		\item{Pesquisa e Desenvolvimento de Hardware e Protótipos de Instrumentos Musicais}
		\item{Título do Projeto: \textit{Novas Interfaces para Aplicações de Música e Áudio}}
		\item{Bolsista da FACEPE BCT-0103-1.03/17}
	\end{cvitems}}

%---------------------------------------------------------
%---------------------------------------------------------
\cventry
	{Pesquisador e Desenvolvedor}
	{Projeto de Pesquisa em Dança - Funcultura}
	{Recife, Brazil}
	{Mar. 2017 - Mar. 2018}
	{\begin{cvitems}
		\item{Desenvolvimento de Sistema Interativo para Dança e Música}
		\item{Parceria com músico, ator e dançarino Helder Vasconcelos}
		\item{Projeto incentivado pelo Funcultura, Pernambuco}
		\item{Título do Projeto: \textit{Princípios da Tradição Popular no Contemporâneo e no Desenvolvimento de Instrumentos Digitais de Música e Dança}}
	\end{cvitems}}

%---------------------------------------------------------
%---------------------------------------------------------
\cventry
	{Produtor Tecnológico}
	{Janeiro de Grandes Espetáculos}
	{Recife, Brasil}
	{Jan. 2018}
	{\begin{cvitems}
		\item{Espetáculo: \textit{Eu Sou} de Helder Vasconcelos}
		\item{Produção dos Instrumentos Digitais}
	\end{cvitems}}

%---------------------------------------------------------
%---------------------------------------------------------
% \cventry
% 	{Professor}
% 	{CESAR.edu}
% 	{Recife, Brasil}
% 	{Out. 2017}
% 	{\begin{cvitems}
% 		\item{Mestrado Profissional em Design}
% 		\item{Disciplina: Cidades Inteligentes e Cidades Lúdicas}
% 	\end{cvitems}}
% 
%---------------------------------------------------------
%---------------------------------------------------------
% \cventry
% 	{Professor}
% 	{CESAR.edu}
% 	{Manaus, Brasil}
% 	{Maio 2017}
% 	{\begin{cvitems}
% 		\item{Mestrado Profissional em Design}
% 		\item{Disciplina: Cidades Inteligentes e Cidades Lúdicas}
% 	\end{cvitems}}
% 
%---------------------------------------------------------
%---------------------------------------------------------
% \cventry
% 	{Professor}
% 	{CESAR.edu}
% 	{Recife, Brasil}
% 	{Set. 2016}
% 	{\begin{cvitems}
% 		\item{Mestrado Profissional em Design}
% 		\item{Disciplina: Cidades Inteligentes e Cidades Lúdicas}
% 	\end{cvitems}}
% 
%---------------------------------------------------------
%---------------------------------------------------------
\cventry
	{Pesquisador Residente Voluntário}
	{L.O.U.Co - Porto Digital}
	{Recife, Brasil}
	{Maio 2016 - Jul. 2016}
	{\begin{cvitems}
		\item{Laboratório de Objetos Urbanos Conectados}
		\item{Pesquisa em Técnicas de Fabricação Digital}
		\item{Auxílio a novos usuários no uso dos equipamentos do laboratório}
	\end{cvitems}}

%---------------------------------------------------------
%---------------------------------------------------------
\cventry
	{Pesquisador visitante}
	{Mjolnir - Inria Lille}
	{Lille, França}
	{Jan. 2016}
	{\begin{cvitems}
		\item{Colaboração científica para concepção de interface de usuário para ambiente de prototipação física}
		\item{Instituto Nacional Francês de Pesquisa em Ciência da Computação e Automação (Inria)}
		\item{Laboratório de Pesquisa em Interação Humano-Computador}
		\item{Orientador: Stéphane Huot}
	\end{cvitems}}

%---------------------------------------------------------
%---------------------------------------------------------
% \cventry
% 	{Professor}
% 	{CESAR.edu}
% 	{Recife, Brasil}
% 	{Nov. 2014}
% 	{\begin{cvitems}
% 		\item{Mestrado Profissional em Design}
% 		\item{Disciplina: Prototipação Eletrônica}
% 	\end{cvitems}}
% 
%---------------------------------------------------------
%---------------------------------------------------------
\cventry
	{Pesquisador e Desenvolvedor}
	{Projeto de Pesquisa em Música - Funcultura}
	{Recife, Brazil}
	{Mar. 2014 - Fev. 2015}
	{\begin{cvitems}
		\item{Desenvolvimento de Instrumentos Musicais Digitais}
		\item{Projeto incentivado pelo Funcultura, Pernambuco}
		\item{Título do Projeto: \textit{Diálogos entre a Lutheria Digital e a Música Popular Pernambucana}}
	\end{cvitems}}

%---------------------------------------------------------
%---------------------------------------------------------
\cventry
	{Professor}
	{Faculdade Boa Viagem}
	{Recife, Brasil}
	{Mar. 2014 - Ago. 2014}
	{\begin{cvitems}
		\item{Curso de Ciência da Computação}
		\item{Disciplinas: Projeto de Desenvolvimento e Sistemas Embarcados}
	\end{cvitems}}

%---------------------------------------------------------
%---------------------------------------------------------
\cventry
	{Professor}
	{CESAR.edu}
	{Recife, Brasil}
	{Jul. 2013}
	{\begin{cvitems}
		\item{Especialização em Design de Interação}
		\item{Disciplina: Prototipação Eletrônica}
	\end{cvitems}}

%---------------------------------------------------------
%---------------------------------------------------------
\cventry
	{Pesquisador e Desenvolvedor}
	{Pesquisa em Dança e Tecnologia}
	{Recife, Brasil}
	{Jan. 2012 - Jan. 2014}
	{\begin{cvitems}
		\item{Desenvolvimento de Instrumentos Musicais de Música e Dança}
		\item{Título do Projeto: \textit{Foco}}
		\item{Projeto de Helder Vasconcelos aprovado no Rumos Itaú Cultural 2012-2014}
	\end{cvitems}}

%---------------------------------------------------------
%---------------------------------------------------------
\cventry
	{Engenheiro de Software}
	{Cápsula Inovação em Tecnologia da Informação Ltda.}
	{Recife, Brazil}
	{Mar. 2010 - Dez. 2010}
	{\begin{cvitems}
		\item{Funcionário em Regime CLT}
		\item{Desenvolvimento de Plataforma Web em GWT}
	\end{cvitems}}

%---------------------------------------------------------
%---------------------------------------------------------
% \cventry
% 	{Instrutor}
% 	{Projeto de Extensão para Funcionários da UFPE (CIn-UFPE)}
% 	{Recife, Brasil}
% 	{Abr. 2008 - Maio 2008}
% 	{\begin{cvitems}
% 		\item{Curso de Informática Básica}
% 		\item{Professor Tutor: Fernando da Fonseca de Souza}
% 	\end{cvitems}}

%---------------------------------------------------------
%---------------------------------------------------------
% \cventry
% 	{Monitor Voluntário}
% 	{Monitoria da Disciplina de Informática Teórica (CIn-UFPE)}
% 	{Recife, Brasil}
% 	{Fev. 2008 - Dez. 2008}
% 	{\begin{cvitems}
% 		\item{Graduação em Ciência da Computação}
% 		\item{Professores: Ruy José Guerra Barretto de Queiroz e Anjolina Grisi de Oliveira}
% 	\end{cvitems}}

%---------------------------------------------------------
%---------------------------------------------------------
\cventry
	{Aluno Voluntário}
	{Iniciação Científica em Redes de Computadores (CIn-UFPE)}
	{Recife, Brasil}
	{Ago. 2007 - Jul. 2008}
	{\begin{cvitems}
		\item{Orientador: Paulo André da Silva Gonçalves}
		\item{Título do Projeto: \textit{Impacto de Condições Adversas de Rede na Transmissão de Vídeo P2P}}
	\end{cvitems}}

%---------------------------------------------------------
%---------------------------------------------------------
\cventry
	{Instrutor}
	{Projeto de Extensão para Funcionários da UFPE (CIn-UFPE)}
	{Recife, Brasil}
	{Maio 2007 - Jul. 2007}
	{\begin{cvitems}
		\item{Curso de Informática Básica}
		\item{Professor Tutor: Fernando da Fonseca de Souza}
	\end{cvitems}}

%---------------------------------------------------------
%---------------------------------------------------------
\cventry
	{Monitor Voluntário}
	{Monitoria da Disciplina de Introdução à Programação (CIn-UFPE)}
	{Recife, Brasil}
	{Fev. 2006 - Dez. 2007}
	{\begin{cvitems}
		\item{Graduação em Ciência da Computação}
		\item{Professor: Paulo Henrique Monteiro Borba}
	\end{cvitems}}

%---------------------------------------------------------
\end{cventries}
